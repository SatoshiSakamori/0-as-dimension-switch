\documentclass[12pt,a4paper]{article}
\usepackage{amsmath,amssymb,amsthm}
\usepackage{bm}
\usepackage{geometry}
\geometry{margin=25mm}
\usepackage{hyperref}

\title{Zero as a Dimension Switch:\\
A Conceptual Framework for Reinterpreting the Number Line}
\author{Satoshi Sakamori}
\date{\today}

\begin{document}
\maketitle

\begin{abstract}
In the conventional real number line, zero is positioned at the center between positive and negative values. 
However, zero exhibits fundamentally different properties from other numbers: division by zero is undefined, and it often marks the location of singularities. 
This paper proposes a reinterpretation of zero as a \emph{dimension switch} --- a non-numeric boundary between two distinct spaces, thereby avoiding certain divergences and redefining the continuity around zero.
\end{abstract}

\section{Introduction}
In standard arithmetic, zero acts as the additive identity and lies at the center of symmetry in the real number line. 
Despite this central role, operations involving zero, such as division by zero, remain undefined. 
This conceptual framework explores treating zero not as a number but as a transition point between two dimensions.

\section{Motivation}
Two main issues motivate this reinterpretation:
\begin{enumerate}
    \item \textbf{Singularities:} Functions such as $f(x) = 1/x$ diverge at $x = 0$.
    \item \textbf{Physical Analogies:} In thermodynamics, absolute zero represents a one-sided boundary; temperatures do not extend below $0$ K in the classical sense.
\end{enumerate}
By redefining zero as a structural boundary, these difficulties may be reframed rather than patched within existing frameworks.

\section{Definition of Spaces}
We define:
\begin{itemize}
    \item $\mathbb{R}^+$: the \emph{positive space}, containing numbers such as $1, 2, \pi, \dots$
    \item $\mathbb{R}^-$: the \emph{mirror space}, containing elements $1^{-}, 2^{-}, \pi^{-}, \dots$
    \item $0$: a non-numeric \emph{switch point} between the two spaces
\end{itemize}

The overall structure can be expressed as:
\[
    \mathbb{R} = \mathbb{R}^+ \ \cup\ \{0\} \ \cup\ \mathbb{R}^-
\]

Here, $\mathbb{R}^-$ is not merely the set of numbers less than zero; it is an entirely separate dimension with symbolic elements.

\section{Transformation Across Zero}
We introduce a \emph{space transformation}:
\[
    T: \mathbb{R}^+ \to \mathbb{R}^-, \quad T(a) = a^{-}
\]
and its inverse
\[
    T^{-1}: \mathbb{R}^- \to \mathbb{R}^+, \quad T^{-1}(a^{-}) = a
\]
These transformations are not arithmetic inverses; they represent a conceptual translation across the zero boundary.

\section{Behavior Near Zero}
In this framework, there is no requirement for continuity across zero:
\[
    \lim_{x \to 0^+} f(x) \ \neq \ \lim_{x \to 0^-} f(x)
\]
The right-hand limit is taken in $\mathbb{R}^+$ and the left-hand limit in $\mathbb{R}^-$, and they may belong to incompatible domains.

This separation allows for reinterpretation of problematic expressions like $1/x$ at $x = 0$ as involving a change of dimension rather than a blow-up to infinity.

\section{Potential Applications}
Potential mathematical and physical applications include:
\begin{itemize}
    \item Reinterpreting singularities in rational functions
    \item Modeling one-sided physical quantities (e.g., absolute temperature)
    \item Exploring connections to projective geometry and Riemann surfaces
    \item Developing algebraic systems with multiple interconnected spaces
\end{itemize}

\section{Future Work and Open Questions}
Future developments may include:
\begin{itemize}
    \item Rigorous algebraic formalization
    \item Geometric representation using animation tools such as Manim
    \item Computational simulations to explore functional behavior under $T$
    \item Extensions to higher-dimensional spaces and non-commutative algebra
\end{itemize}

\section*{Acknowledgements}
The author thanks language and structure assistance provided by large language models during the drafting process.

\end{document}